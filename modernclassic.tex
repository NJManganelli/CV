%%%%%%%%%%%%%%%%%%%%%%%%%%%%%%%%%%%%%%%%%
% "ModernCV" CV and Cover Letter
% LaTeX Template
% Version 1.3 (29/10/16)
%
% This template has been downloaded from:
% http://www.LaTeXTemplates.com
%
% Original author:
% Xavier Danaux (xdanaux@gmail.com) with modifications by:
% Vel (vel@latextemplates.com)
%
% License:
% CC BY-NC-SA 3.0 (http://creativecommons.org/licenses/by-nc-sa/3.0/)
%
% Important note:
% This template requires the moderncv.cls and .sty files to be in the same 
% directory as this .tex file. These files provide the resume style and themes 
% used for structuring the document.
%
%%%%%%%%%%%%%%%%%%%%%%%%%%%%%%%%%%%%%%%%%

%----------------------------------------------------------------------------------------
%	PACKAGES AND OTHER DOCUMENT CONFIGURATIONS
%----------------------------------------------------------------------------------------

\documentclass[10pt,a4paper,sans]{moderncv} % Font sizes: 10, 11, or 12; paper sizes: a4paper, letterpaper, a5paper, legalpaper, executivepaper or landscape; font families: sans or roman

\moderncvstyle{classic} % CV theme - options include: 'casual' (default), 'classic', 'oldstyle' and 'banking'
\moderncvcolor{blue} % CV color - options include: 'blue' (default), 'orange', 'green', 'red', 'purple', 'grey' and 'black'

\usepackage[scale=0.85]{geometry} % Reduce document margins
%\setlength{\hintscolumnwidth}{3cm} % Uncomment to change the width of the dates column
%\setlength{\makecvtitlenamewidth}{10cm} % For the 'classic' style, uncomment to adjust the width of the space allocated to your name

%----------------------------------------------------------------------------------------
%	NAME AND CONTACT INFORMATION SECTION
%----------------------------------------------------------------------------------------
\firstname{Nick} % Your first name
\familyname{Manganelli} % Your last name

% All information in this block is optional, comment out any lines you don't need
\title{Curriculum Vitae}
%\address{123 Broadway}{City, State 12345}
\mobile{+1 (623) 760 7276}
%\phone{(000) 111 1112}
%\fax{(000) 111 1113}
\email{Nick.Manganelli@gmail.com}
%\homepage{staff.org.edu/~jsmith}{staff.org.edu/$\sim$jsmith} % The first argument is the url for the clickable link, the second argument is the url displayed in the template - this allows special characters to be displayed such as the tilde in this example
%\extrainfo{additional information}
%\photo[70pt][0.4pt]{pictures/picture} % The first bracket is the picture height, the second is the thickness of the frame around the picture (0pt for no frame)
%\quote{"A witty and playful quotation" - John Smith}

%----------------------------------------------------------------------------------------

\begin{document}

%----------------------------------------------------------------------------------------
%	COVER LETTER
%----------------------------------------------------------------------------------------

% To remove the cover letter, comment out this entire block

%\clearpage

%\recipient{HR Department}{Corporation\\123 Pleasant Lane\\12345 City, State} % Letter recipient
%\date{\today} % Letter date
%\opening{Dear Sir or Madam,} % Opening greeting
%\closing{Sincerely yours,} % Closing phrase
%\enclosure[Attached]{curriculum vit\ae{}} % List of enclosed documents

%\makelettertitle % Print letter title


%\makeletterclosing % Print letter signature

%\newpage

%----------------------------------------------------------------------------------------
%	CURRICULUM VITAE
%----------------------------------------------------------------------------------------

\makecvtitle % Print the CV title

%----------------------------------------------------------------------------------------
%	EDUCATION SECTION
%----------------------------------------------------------------------------------------
\section{Ph.D. Dissertation}

\cvitem{Title}{\emph{Search for Standard Model Production of Four Top Quarks in Proton-Proton Collisions at 13 TeV in the Opposite-Sign Dilepton Channel Using CMS Data From 2017 and 2018}}
\cvitem{Supervisors}{Professor Robert Clare \& Professor Steve Wimpenny}
%\cvitem{Description}{Description}

\section{Education}

\cventry{Sep 2022}{Ph.D. in Physics}{The University of California}{Riverside}{}{}  % Arguments not required can be left empty
\cventry{Dec 2017}{Master of Science in Physics}{The University of California}{Riverside}{\textit{GPA -- 3.9}}{}
\cventry{May 2012}{Bachelor of Science in Physics}{Northern Arizona University}{}{\textit{GPA -- 4.0}}{Magna Cum Laude}
\cventry{May 2012}{Bachelor of Science in Mathematics}{Northern Arizona University}{}{\textit{GPA -- 4.0}}{Magna Cum Laude}


%----------------------------------------------------------------------------------------
%	AWARDS SECTION
%----------------------------------------------------------------------------------------

\section{Awards}
%\cvitemwithcomment{2021}{Certificate of Excellence}{\textit{Seventh Summer School on Machine Learning in HEP}} %In July
\cvitem{2021}{Certificate of Excellence, Seventh Summer School on Machine Learning in HEP}
\cvitem{2019}{CMS Achievement Award, Compact Muon Solenoid Collaboration}
%\cvitemwithcomment{2017}{Provost Research Fellowship}{\textit{University of California - Riverside}}%%_Nomination_ for outstanding teaching award
\cvitem{2016}{Provost Research Fellowship, University of California - Riverside}
\cvitem{2010}{Adel Physics Scholarship, Northern Arizona University}
\cvitem{2009 -- 2012}{Dean's List, Northern Arizona University}
%----------------------------------------------------------------------------------------
%	SKILLS SECTION
%----------------------------------------------------------------------------------------
\section{Hard Skills}
\cvitem{Programming}{\textsc{Python} (Advanced), \textsc{C++} (Advanced), \textsc{Python Packaging}, \textsc{Xilinx HLS (Field Programmable Gate Arrays)}, \LaTeX, \textsc{zsh}, \textsc{Java} (Basic)}
\cvitem{Tools}{Git, PyTorch (Artificial Intelligence, ML), Scikit-learn (ML), Scikit-hep (uproot, awkward), coffea, RDataFrame}
\cvitem{Software}{CMS Software (CMSSW), \textsc{CUDA (GPUs)}, \textsc{ROOT}}
\cvitem{Teaching}{Lecturer/Instructor for programming schools and tutorials, physics labs and discussion sections}
\cvitem{Training}{Cherry-picker operation, Working at heights, 
Work in radiation-control zones}

\section{Soft Skills}
\cvitem{Communication}{Presentations, Public Speaking, Large Group Coordination}
\cvitem{Leadership}{Management, Supervision, Mentoring, Delegation, Planning, Scheduling, Collaboration}
\cvitem{Languages}{English (Mothertongue), Spanish (Basic), French (Basic)}


%\cvitem{Soft Skills}{Team Leadership, Mentoring, 
%Coordination work}
%----------------------------------------------------------------------------------------
%	WORK EXPERIENCE SECTION
%----------------------------------------------------------------------------------------
\clearpage

%----------------------------------------------------------------------------------------
%	PUBLICATIONS SECTION
%----------------------------------------------------------------------------------------
\section{Peer Reviewed Publications (Primary or Key Author)}
\cventry{Jul 2024}{Savard, C., Manganelli, N., Holzman, B. et al.}{Optimizing High Throughput Inference on Graph Neural Networks at Shared Computing Facilities with the NVIDIA Triton Inference Server}{\href{https://doi.org/10.1007/s41781-024-00123-2}{10.1007/s41781-024-00123-2}}{Computing and Software for Big Science 8, 14 (2024)}{}
\cventry{Sep 2023}{CMS Collaboration (Primary Author)}{Evidence for four-top quark production in proton-proton collisions at $\sqrt{s} = 13TeV$}{\href{https://doi.org/10.1016/j.physletb.2023.138076}{10.1016/j.physletb.2023.138076}}{\textit{Physics Letters B, Volume 844, Sep 10, 2023 (138076)}}{}
%%\bibitem{fourtops}{A.~Tumasyan \textit{et al.} [CMS Collaboration]. ``Evidence for four-top quark production in proton-proton collisions at s=13TeV.'' {\em Physics Letters B}. \textbf{844} pp. 138076 (2023), https://doi.org/10.1016/j.physletb.2023.138076}
\cventry{Aug 2022}{IRIS-HEP (Key Author)}{HSF IRIS-HEP Second Analysis Ecosystem Workshop Report}{From the AES II Workshop in Paris}{\href{https://doi.org/10.5281/zenodo.7003963}{https://doi.org/10.5281/zenodo.7003963}}{}
\cventry{Mar 2020}{Nick Manganelli}{Upgrades to the CMS Cathode Strip Chambers for the HL-LHC}{Presented at Innovative Particle and Radiation Detectors 2019}{\textit{Journal of Instrumentation, Volume 15, March 2020 (C03047)}}{}

\section{Advanced Topics Instructor at CERN and Fermilab}
%\cventry{}{}{}{}{}{CERN operates on the cutting edge of science, and many advanced skills are not taught in physics and computer science graduate programs. In particular, experience with analyzing Big Data and applying advanced Machine Learning are essential for new students and researchers. Accordingly, individuals with recognized expertise in these areas are invited to provide special enrichment courses at workshops and schools, and I was honored to be chosen to present classes at 5 such events:}
\cventry{}{\normalfont{CERN operates on the cutting edge of science, and many advanced skills are not taught in physics graduate programs. In particular, experience with analyzing Big Data and applying advanced Machine Learning are essential for new students and researchers. Accordingly, individuals with recognized expertise in these areas are invited to provide special enrichment courses at workshops and schools, and I was honored to be chosen to present classes at multiple such events:}}{}{}{}{}

%\section{Schools, Tutorials}
%%%\cventry{Jun 2024}{Big Data Analysis at CERN}{USCMS/IRIS-HEP Analysis Software Training}{\href{https://indico.cern.ch/event/1383972/contributions/5825305/}{coffea lecture} \href{https://indico.cern.ch/event/1383972/contributions/5825307/}{and exercise} \href{https://indico.cern.ch/event/1383972/}{, https://indico.cern.ch/event/1383972/}}{}{}
\cventry{Jul 2024}{Scikit-HEP Instructor}{CMS HATS @ LPC 2024, coffea}{\href{https://indico.cern.ch/event/1433935/}{https://indico.cern.ch/event/1433935/}}{}{}
\cventry{Jun 2024}{Big Data Analysis at CERN}{USCMS/IRIS-HEP Analysis Software Training}{\href{https://indico.cern.ch/event/1383972/contributions/5825305/}{coffea lecture} \href{https://indico.cern.ch/event/1383972/contributions/5825307/}{and exercise} \href{https://indico.cern.ch/event/1383972/}{, https://indico.cern.ch/event/1383972/}}{}{}
\cventry{Jan 2024}{CMS Data Analysis School Facilitator}{CMS Data Analysis School @ LPC 2024, TTGamma Long Exercise}{\href{https://indico.cern.ch/event/1333922/timetable/\#b-531287-parallel-session-long}{https://indico.cern.ch/event/1333922/timetable/}}{}{}
\cventry{Dec 2023}{Machine Learning Instructor}{2nd COFI Advanced Instrumentation and Analysis Techniques School}{\href{https://indico.cern.ch/event/1299889/timetable/}{Tools and Resources Needed for Machine Learning, https://indico.cern.ch/event/1299889/timetable/}}{}{}
\cventry{Jul 2023}{Scikit-HEP Instructor}{CMS HATS @ LPC 2023, uproot and awkward-array}{\href{https://indico.cern.ch/event/1297663/}{https://indico.cern.ch/event/1297663/}}{}{}
%%\bibitem{cofimltools}{Manganelli, N. ``Tools and Resources Needed for Machine Learning.'' Lecture, {\em 2nd COFI Advanced Instrumentation and Analysis Techniques School}, San Juan, Puerto Rico, December 9-10, 2023.}
\cventry{Aug 2022}{Scikit-HEP Instructor}{CMS HATS @ LPC 2022, uproot and awkward-array}{\href{https://indico.cern.ch/event/1186603/}{https://indico.cern.ch/event/1186603/}}{}{}%{By request of HATS Organizers}{}
%%%\cventry{Jan 2021}{CERN CMS Data Analysis School Facilitator}{CMS Data Analysis School @ LPC 2021, Top Mass Exercise}{\href{https://indico.cern.ch/event/966368/timetable/}{https://indico.cern.ch/event/966368/timetable/}}{}{}
\cventry{Jan 2021}{CMS Data Analysis School Facilitator}{CMS Data Analysis School @ LPC 2021, Top Mass Exercise}{\href{https://indico.cern.ch/event/966368/timetable/}{https://indico.cern.ch/event/966368/timetable/}}{}{}

\clearpage
\section{Research and Work Experience}
%====================
% RESEARCH EXPERIENCE
%====================
\cventry{Jan 2023 -- Now}{Postdoctoral Researcher}{University of Colorado, Boulder}{Fermilab, Batavia, IL, USA}{}{Analysis searching for Electroweak SUSY at the LHC using the CMS Detector's 2017-2023 data.
\begin{itemize}
\item Based at Fermilab
\item (WIP) Conducting an analysis searching for Electroweak SUSY in the  $ZH + MET$ channel at the LHC using the CMS Detector's 2017-2023 data.
\item Designed cutting-edge containerized framework to support analysis, using the latest versions of software from the IRIS-HEP and HSF groups as well as industry: coffea, scikit-hep (awkward, uproot, vector),  dask
\item Conducted initial feasibility studies for the analysis, ensuring good sensitivity and room to improve over previous attempts in CMS and ATLAS
\item Prepared Monte Carlo request for new simulation samples based on my sensitivity studies
\item Preparing to do studies within the analysis of Full Simulation against Fast Simulation and the new Flash Simulation methods for computationally-efficient Monte Carlo production within CMS
\newline{}
\end{itemize}
Performed work for the CMS Phase II Level 1 Trigger Upgrade
\begin{itemize}
\item Maintainer of CMSSW emulation (C++) and firmware (HLS) for the Global Track Trigger upgrade
\item Attained bit-accurate agreement between emulation and HLS firmware for the central GTT algorithm, FastHisto, which is responsible for finding the Primary Vertex in a collision and permits PUPPI reconstruction at Level 1 in the Correlator Trigger
\item Mentored and lead 5 graduate students on their algorithm development within GTT, for Track Jets, Track MET, Track HT, and Meson reconstruction
\item Reviewed code for above algorithms and provided feedback
\item Refactored and updated emulation code to improve writing and reading raw data, essential ingredients for CMSSW data packers and unpackers
\item Fixed identified bugs and improved code for better maintainability across the GTT emulation stack, following the latest developments and best practices
\item Liased between CMS L1T Offline Software and GTT group to facilitate faster and more efficient work
\item Spearheaded and developed code for GTT physics objects in the newly created L1Nano central data format, beginning much-needed consolidation of the analysis format and paving the way to more shareable, centrally-validated workflows
\item Engaged with the community through questions and feedback in discussions to contribute to the CMS experiment's data analysis tools and methods
\newline{}
\end{itemize}
}

%====================
% RESEARCH EXPERIENCE
%====================
\cventry{Oct 2023 -- Now}{Software R\&D Lead, Column-Joining with ServiceX and Scikit-HEP}{University of Colorado, Boulder}{Fermilab, Batavia, IL, USA}{}{US-CMS funded project to perform columnar data-joining
\begin{itemize}
\item Wrote proposal for and awarded funding for project (half-salary renumeration)
\item Ongoing project to combine industry and HEP-specific SW to solve challenges with limited disk space and compute power for the upcoming High-Luminosity LHC era, in which 10x the LHC data is expected
\item Collaborated with Ben Galewsky (Sr. Research Software Engineer at National Center for Supercomputing Applications) and Burt Holzman (Fermi National Accelerator Laboratory)
\item Prototyped components for an end-to-end data-transformation and joining system leveraging ServiceX, Trino, and coffea
\item Benchmarked code to identify bottlenecks, ensuring viability of design
\item (WIP) Designing code to build TypeTracers for joinable data in ROOT format to then feed through a coffea and scikit-hep analysis, triggering on-demand column-joining
\item (WIP) Writing code to bridge to ServiceX to handle the transformation of ROOT to parquet
\item (WIP) Writing code to automatically deduce and submit a query to the distributed Trino service, ingesting parquet data and efficiently joining only the necessary columns for an analysis
\newline{}
\end{itemize}}

%====================
% RESEARCH EXPERIENCE
%====================
\cventry{March 2023 -- June 2024}{BTV Contact for CMS SUS PAG}{University of Colorado, Boulder}{Fermilab, Batavia, IL, USA}{}{
\begin{itemize}
\item Served as main point of contact for analyses in the CMS SUS Physics Analysis Group
\item Mentored analysis groups in understanding and implementing the recommendations of the B-tagging and Vertexing Physics Object Group
\item Provided review and feedback promptly
\item Reviewed more than a dozen analyses for compliance with recommendations; guided updates to and approval of b-tagging usage
\newline{}
\end{itemize}}

%====================
% RESEARCH EXPERIENCE
%====================
\cventry{Jan 2023 -- Dec 2023}{Software R\&D Lead, Columnar Analysis and Inference as a Service}{University of Colorado, Boulder}{Fermilab, Batavia, IL, USA}{}{US-CMS funded project to integrate Inference-as-a-Service with HEP analysis tools
\begin{itemize}
\item Assumed role of project lead
\item Designed benchmarking code for various scenarios, as well as Prometheus metric analysis
\item Supervised graduate student in conducting benchmark tests
\item Analyzed data to identify system bottlenecks and improve performance of the system at a premier DOE lab
\item Paper accepted for publication in Computing and Software for Big Science, pre-print available on arXiv
\newline{}
\end{itemize}}

%====================
% RESEARCH EXPERIENCE
%====================
\cventry{Apr 2018 -- Sep 2022}{Graduate Student Researcher}{University of California, Riverside}{CERN, Geneva, CH}{}{Performed dissertation work searching for the production of $t\bar{t}t\bar{t}$ at the LHC using the CMS Detector's 2017-2018 data.
\begin{itemize}
\item Posted at CERN for hardware work and physics analysis
\item Performed studies evaluating new resolved top-tagging algorithms for identification of jets from top decays
\item Built new analysis framework using ROOT's RDataFrame, leveraging Python and C++
\item Implemented entirety of trigger selection and channel orthogonality, POG corrections, systematic uncertainties, and conversion to CMS's statistical inference package (Higgs Combine)
\item Performed systematic uncertainty studies, and Monte Carlo sample stitching for two channels of the CMS Four Top combination (CADI TOP-21-005)
\item Took on role of postdoctoral researcher and mentored junior graduate student in the group on basic analysis and coding practices, ROOT, and CMS Software stack (CMSSW)
\newline{}
\end{itemize}}

%====================
% RESEARCH EXPERIENCE
%====================
\cventry{Sep 2019 -- Oct 2021}{CSC Deputy Run Coordinator}{CMS Collaboration}{CERN, Geneva, CH}{}{Commissioned chambers, supervised Detector On Call shifters, and coordinated with CMS for the preparation of the detector for data taking in 2022.
\begin{itemize}
\item Worked alongside CSC Run Coordinator and with Technical Coordinators
\item Served as liaison with the rest of CMS Run Coordination for daily operations
\item Crucial role in commissioning newly upgraded and re-installed chambers
\item Coordinated tests with new GEM sub-detector operations team
\item Trained and supervised students and postdocs working as Detector On Call shifters
\item Debugged detector problems by fixing connections and replacing electronics on the CMS detector, employing a cherry-picker for work on chambers
\item Coordinated activities related to commissioning software and firmware upgrades
\item Served as Acting Run Coordinator for last 2 months of my 2 year term, and trained in-coming Deputy Run Coordinator
\newline{}
\end{itemize}}

%====================
% RESEARCH EXPERIENCE
%====================
\cventry{Nov 2018 -- Sep 2019}{Test Team Leader, Member}{CMS Collaboration}{CERN, Geneva, CH}{}{Lead testing of upgraded Cathode Strip Chambers, the 'Critical Path' work of the CMS collaboration during the LHC shutdown period between mid 2018 and early 2022.
\begin{itemize}
\item Tested and debugged electronics and chamber problems after their extraction and electronics upgrades
\item Followed controlled radiation zone procedures during work with irradiated chambers that were extracted from the CMS detector
\item Lead one of two test teams for latter half of the role, overseeing several graduate students and postdoctoral researchers
\item Assisted with test procedures and documentation
\item Brainstormed solutions to various problems
\item Ensured on-time delivery of upgraded chambers for CMS's Muon Critical Path work
\newline{}
\end{itemize}}

%====================
% RESEARCH EXPERIENCE
%====================
\cventry{Jul 2018 -- Oct 2018}{Test Stand Construction}{CMS Collaboration}{CERN, Geneva, CH}{}{Supported Upgrades of the Large Hadron Collider's Compact Muon Solenoid experiment.
\begin{itemize}
\item Worked in tandem with Staff Scientist from Cathode Strip Chamber sub-detector
\item Set up server blades with new operating systems
\item Installed upgraded electronics on chambers
\item Debugged and upgraded test software to coincide with new FPGA
firmware
\item Provided feedback for experts to iterate on software and firmware
\item Successfully completed the construction of 3 test stands for the CMS critical-path upgrades to muon endcap
\newline{}
\end{itemize}}


%====================
% RESEARCH EXPERIENCE
%====================
\cventry{Jul 2016 -- Sep 2016}{Provost Fellowship Researcher}{University of California, Riverside}{Riverside, California}{}{Summer Research Program in Heavy Ion Physics.
\begin{itemize}
\item Supervised by Prof. Richard Seto
\item 8 week summer research for the Relativistic Heavy Ion Collider
\item Conducted literary review of then-current research on Quark Gluon Plasma, notably on lenticular flow and multi-particle correlations as signs of the QGP state of matter
\item Prototyped C++ code to reconstruct muon showers in the Muon Piston Calorimeter for the Phenix detector over the course of 5 weeks
\newline{}
\end{itemize}}

%====================
% INDUSTRY EXPERIENCE
%====================
\cventry{2012 -- 2016}{VP}{A.T.L. Industries}{Camp Verde, AZ}{}{Served as executive in a small family-owned business manufacturing specialized tools for the U.S. federal government and commercial trucking industry.
\begin{itemize}
\item Worked with suppliers to ensure timely delivery of raw materials
\item Negotiated purchase agreements with S.P.X. Corporation and parent company Bosch
\item Outsourced specialized component production to highly qualified contract suppliers
\newline{}
\end{itemize}}

%====================
% UNDERGRAD EXP
%====================
\cventry{2011 -- 2012}{Undergraduate Student Researcher}{Northern Arizona University}{Flagstaff, AZ}{}{
\begin{itemize}
\item Supervised by Professor Mark James
\item Used Matlab to create simulation code for propagation of laser light and reflection from non-planar surfaces
\newline{}
\end{itemize}}

%====================
% UNDERGRAD EXP
%====================
\cventry{2010 --- 2011}{Undergraduate Student Researcher}{Northern Arizona University}{Flagstaff, AZ}{}{
\begin{itemize}
\item Supervised by Professor David Cornelison
\item Used infrared spectroscopy to study astrophysical ices
\item Constructed new sample cells for use in high-vacuum environment
\item Upgraded equipment, such as adding new vacuum turbofans
\newline{}
\end{itemize}}


\section{University Teaching Experience}
%====================
% TEACHING EXPERIENCE
%====================
\cventry{Jan 2018 -- Mar 2018}{Discussion Section Leader}{University of California, Riverside}{Riverside, CA}{}{Supplementary physics instruction for Life Science Majors.
\begin{itemize}
\item Supplementary instructor for introductory physics
\item Lead discussion sections
\item Assisted students in understanding physics concepts using instruction complementary to their main lectures
\newline{}
\end{itemize}}

%====================
% TEACHING EXPERIENCE
%====================
\cventry{Sep 2017 -- Dec 2017}{Lab Teaching Assistant}{University of California, Riverside}{Riverside, CA}{}{Introductory laboratory for Science and Engineering Majors.
\begin{itemize}
\item Guided students in introductory physics lab experiments
\item Graded lab reports, mid-terms, and finals
\item Crafted weekly quizzes to test students' learning progress.
\item Helped students understand physics concepts using a variety of teaching techniques
\newline{}
\end{itemize}}


%----------------------------------------------------------------------------------------
%	PAPERS IN PROGRESS SECTION
%----------------------------------------------------------------------------------------
% FIXME: Triton Inference Paper
%%\bibitem{tritonpaper}{C.~Savard, N.~Manganelli, B.~Holtzman, L.~Gray. K.~Stenson, K.~Ulmer, K.~Pedro, A.~Perloff ``Optimizing High Throughput Inference on Graph Neural Networks at Multi-User Computing Facilities with the NVIDIA Triton Inference Server.'’ {Being submitted to Computing and Software for Big Science}}
%----------------------------------------------------------------------------------------
%	ACTIVITIES SECTION
%----------------------------------------------------------------------------------------

%----------------------------------------------------------------------------------------
%	TUTORIALS SECTION
%----------------------------------------------------------------------------------------
%\section{Tutorials}
%%\bibitem{akhats2023}{Manganelli, N. ``uproot and awkward-array.'' Lecture, {\em CMS HATS @ LPC 2023}, Fermilab, July 6, 2023.}

%----------------------------------------------------------------------------------------
%	CONFERENCE TALKS, WORKSHOPS, SCHOOLS SECTION
%----------------------------------------------------------------------------------------
\clearpage
\section{Conferences, Workshops, Research Talks}%, Workshops, Schools}
\cventry{Jun 2024}{USCMS}{Benchmarking Scalable Triton Inference at Fermilab's Elastic Analysis Facility}{Plenary Talk}{\href{https://indico.cern.ch/event/1352950/contributions/5922016/}{Slides}}{}
\cventry{Jun 2024}{LHCP2024}{Four Top Searches and Constraints on the Top Yukawa in CMS and ATLAS}{Talk}{\href{https://indico.cern.ch/event/1253590/contributions/5928238/}{Slides}}{}
\cventry{Sep 2023}{Level 1 Trigger Workshop}{\textit{Global Track Trigger: Firmware and Algorithm status}}{Talk}{\href{https://indico.cern.ch/event/1288569/contributions/5500407/}{Slides and Recording}}{}
\cventry{Apr 2023}{HL-LHC R\&D Initiative Meeting}{\textit{Triton Inference and Columnar Analysis}}{Talk}{\href{https://indico.cern.ch/event/1272845/}{Link}}{}
\href{https://indico.cern.ch/event/1272845/}{}
\cventry{Sep 2022}{TOP2022}{\textit{Evidence for Four-Top Quark Production at the LHC}}{First Public Results}{}{}
%\cventry{Sep 2022}{TOP2022}{\textit{Evidence for the production of four top quarks in the CMS Run 2 dataset}}{}{}{}
\cventry{Sep 2022}{CERN}{\textit{Ph.D Dissertation Defense}}{}{}{}
\cventry{May 2022}{ROOT Users Workshop}{\textit{12345: Lessons Learned building an Analysis Framework around RDataFrame and CMS NanoAOD}}{Invited Talk}{\href{https://indico.fnal.gov/event/23628/contributions/240753/}{Recording Available Here}}{}

\cventry{May 2022}{Workshop Participant}{Analysis Ecosystems Workshop II}{\href{https://indico.cern.ch/event/1125222/}{https://indico.cern.ch/event/1125222/}}{}{}

\cventry{Apr 2022}{APS April Meeting}{\textit{Search for four-top production in CMS using the latest collision data}}{Followup/Update Talk}{}{}

\cventry{Jul 2021}{Student}{Yandex' Seventh Summer School on Machine Learning in High Energy Physics}{}{}{}

\cventry{Apr 2021}{APS April Meeting}{\textit{Search for four-top production in CMS using the latest collision data}}{}{}{}

\cventry{Oct 2019}{IPRD2019}{\textit{Upgrade to the CMS Cathode-Strip-Muon System for the HL-LHC}}{}{}{}

\cventry{Sep 2018}{CERN}{Ph.D. Candidacy Talk}{}{}{}
%----------------------------------------------------------------------------------------
%	INTERESTS SECTION
%----------------------------------------------------------------------------------------

%\section{Interests}

%\renewcommand{\listitemsymbol}{-~} % Changes the symbol used for lists

%\cvlistdoubleitem{Piano}{Chess}
%\cvlistdoubleitem{Cooking}{Dancing}
%\cvlistitem{Running}

%----------------------------------------------------------------------------------------

%----------------------------------------------------------------------------------------
%	References
%----------------------------------------------------------------------------------------
%\clearpage
%\section{References}
%\cventry{}{Robert Clare}{}{Professor Emeritus of Physics}{}{University of California,Riverside (robert.clare@ucr.edu)}
%\cventry{}{Steve Wimpenny}{}{Professor of Physics}{}{University of California,Riverside (stephen.wimpenny@ucr.edu)}
%\cventry{}{Darien Wood}{}{Professor of Physics}{}{Northeastern University (d.wood@northeastern.edu)}

\end{document}